1、解析ip,将域名与ip地址配对注册
2、登录服务器
3、cd /usr/local/src/  进入此路径下下载nodejs
4、在CentOS 7服务器上安装Node.js.
(wget http://nodejs.org/dist/v8.11.4/node-v8.11.4.tar.gz),wget 后面的链接是网上搜在centos 服务器上安装nodejs的安装包上下载的
https://nodejs.org/en/download/   node官网上链接复制下来到云服务器上安装(详情见图片)

4、ls(下载完可以查看一下)
5、下载下来的node是个node-v8.11.4-linux-x64.tar.xz,是一个压缩包来的,所以接下来我们就是要解压这个包
6、解压包,在服务器上输入  tar -xf node-v8.11.4-linux-x64.tar.xz
7、ls 查看可看到两个文件分别为:
    node-v8.11.4-linux-x64.tar.xz       node-v8.11.4-linux-x64
8、cd node-v8.11.4-linux-x64/----进入node-v8.11.4-linux-x64
9、ls -----查看node-v8.11.4-linux-x64内文件可看到 bin  CHANGELOG.md   include   lib   LICENSE  README.md  share
10、cd bin  -----进入bin文件路径
11、ls  -----查看 bin 文件夹内含有内容有:node  npm  npx
12、./node -v   ------查看node版本为:v8.11.4
13、下载后的node 需要配置文件才能使用,如没有配置文件,我们也可以用以下方式来调用node:
    /usr/local/src/node-v8.11.4-linux-x64/
14、接下来就是将我们本机上编写的web文件在后端复制传给服务器
    后端复制文件给服务器的命令:scp -r ~/Desktop/my_web/ root@134.175.34.221:/usr/local/src

15、然后我们打开服务器,然后直接进入 cd/usr/local/src
16、ls  ----查看有:
    my_web   node-v8.11.4-linux-x64.tar.xz       node-v8.11.4-linux-x64
17、cd my_web/   ------进入 my_web 文件夹
18、ls   ----查看文件有:
    client README.md  server   (这正是刚刚后端复制到服务器上的 my_web 文件夹内的文件内容)
19、cd server/    ------进入server路径下,接下去会调用node
20、sudo ln -s /usr/local/src/node-v8.11.4-linux-x64/bin/npm /usr/bin/     -------在 server 路径下调用全局node
21、/usr/local/src/node-v8.11.4-linux-x64/bin/node ./server.js       ------调用node来打开后端的server.js
22、前面我们已经调用了全局node了,以后要是准备重新传新页面的话,运行server.js时不用再次全局调用node了,可直接输入node server.js 在服务器上运行。
23、然而server.js在服务器上运行时,客户端才能访问到域名下的网页,如果关闭了在服务器上的运行网页就无法上线;
    所以,我们接下来要下载的这个工具叫做forever
24、forevr :一个简单的CLI工具,用于确保给定节点脚本连续运行(即永远)
25、这个工具我们可以直接到 npm 官网去查看怎么安装,接下来我就直接安装此工具了。
26、sudo npm install forever -g    ------在服务器上面安装forever
27、sudo forever start server.js   -----然后在server路径下永远的运行server.js在服务器上运行时,客户端才能访问到域名下的网页,如果关闭了在服务器上的运行网页就无法上线;
28、forever list        
---然后你可以打开forever list查看一下是否 forever processes 是否运行。成功运行的话会有以下信息:forever processes running.....的一条信息呈现。
---即为成功运行 forever。
29、至此我们已把所有的上线过程完成了!

这一块就是介绍,如果你的网页进行更新了,你想更新线上的页面的一个步骤:
1、首先,我们应该登录服务器,然后先结束我们的 forever 工具的运行     所以在服务器上输入------forever stop 0
2、然后在本机后端将我们更新后的文件复制传给服务器            在后端输入------scp -r ~/Desktop/my_web/ root@134.175.34.221:/usr/local/src
3、
